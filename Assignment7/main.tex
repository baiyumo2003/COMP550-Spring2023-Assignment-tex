\documentclass{article}
\usepackage[utf8]{inputenc}
\usepackage{blindtext}
\usepackage{amssymb}
\usepackage{setspace}
\usepackage{lipsum}
\usepackage{hyperref}
\usepackage{amsmath}
\usepackage[english]{babel}
\usepackage{amsthm}
\newtheorem{theorem}{Theorem}
\usepackage{algpseudocode}
\usepackage{algorithm}
\usepackage{graphicx}
\newtheorem{lemma}[theorem]{Lemma}
\usepackage{listings}

\title{COMP 550 – Algorithms and Analysis}
\author{Spring 2023-Assignment07}
\date{  Issue Date: April 19, 2023 
        $\vspace{0.5cm}$
        \\Due Date: Wednesday, April 26, 11:55 pm
        $\vspace{0.5cm}$
        \\ Marks: 5
        \\
        $\vspace{1cm}$
        \\ Student Name:$\rule[-5pt]{8.2cm}{0.05em}$ 
        $\vspace{1cm}$
        \\ Student PID:$\rule[-5pt]{8.2cm}{0.05em}$ }

\begin{document}

\maketitle
\begin{center}
\textbf{Submission:}
\end{center}
\begin{itemize}
    \item \textbf{You must upload your submission(s) before the deadline in Gradescope.}
    \item \textbf{Please ensure that your answers are within the given space allocated after the question.}
\end{itemize}

$\vspace{0.2cm}$

\begin{center}
\textbf{Rules for ALL HWs}:
\end{center}
You are encouraged to \textit{\underline{\textbf{discuss}}} the homework assignments and study together in groups, but when it comes to formulating/writing solutions \textit{\underline{\textbf{you must work}}} \textit{\underline{\textbf{alone and independently}}}. If required, you should be able to explain your answer clearly to TAs/LAs. Copying homework solutions from another student, from the  Internet, solution sets of friends, or other  sources will be considered cheating and treated accordingly.
\newpage

\section{Design an algorithm }
\doublespacing
\begin{enumerate}
    \item Fix a set of positive integers called denominations $x_1,x_2 \dots x_n$. The problem you want to solve for these denominations is the following: \\
    Given an integer $A$, express it as:
    $$A=\sum_{i=1}^{n}a_ix_i$$
    for some nonnegative integers $a_1,\dots a_n \geq 0$
    \begin{enumerate}
        \item Under which condition on the denominations $x_i$ are you able to do this for all integers $A>0$?
        \newpage
        % \item Suppose that you want, Given $A$, to find the nonnegative $a_i's $ that satisfy $A=\sum_{i=1}^{n}a_ix_i$ and such that the sum of all $a_i's $ is minimal ($\min\sum_{i=1}^{n}a_i$). Design a greedy algorithm for this problem, and provide a set of denominations and an $A$ where your algorithm failed. (Your greedy algorithm may not necessarily solve the problem) 
        % \newpage

        \item  Show that greedy algorithm finds optimum $a_i's $ in the case of the denominations are 1,5,10, and 25, for any amount $A$.(Hint: Try to prove by contradiction)
        \newpage
    \end{enumerate}
    \item We call a sequence of $n$ integers $x_1,\dots, x_n$ valid if each $x_i$ is in $\{1,\dots ,m\}$
    \begin{enumerate}
        \item Give a dynamic programming-based algorithm that takes $n,m$ and "Target" $T$ as input and outputs the number of distinct valid sequences such that $x_1+x_2+\dots x_n=T$. Your algorithm should run in $O(m^2n^2)$
        \newpage
        \item Give an algorithm in part (a) that runs in $O(mn^2)$\\
        (Hint: let $f(s, i)$ denote the number of length-i valid sequences with sum equal to $s$. Consider defining the function $g(s,i):=\sum_{t=1}^{s}f(t,i)$)
        \newpage
        \end{enumerate}
\end{enumerate}

\end{document}